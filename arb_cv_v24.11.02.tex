\documentclass[fontsize=11pt]{article}
\usepackage[english]{babel}
\usepackage[utf8]{inputenc}
\usepackage[T1]{fontenc}
\usepackage{lmodern}
\usepackage[protrusion=true,expansion=true]{microtype}
\usepackage[svgnames]{xcolor}
\usepackage[margin=0.75in]{geometry}
\usepackage{url}
\usepackage{textcomp}
\usepackage{hyperref}

\makeatletter
\def\url@modernstyle{
  \@ifundefined{selectfont}{\def\UrlFont{\sf}}{\def\UrlFont{}}}
\makeatother
\urlstyle{modern}

\frenchspacing
\pagestyle{empty}

\renewcommand{\familydefault}{\sfdefault}

\usepackage{sectsty}
\sectionfont{
  \usefont{OT1}{phv}{b}{n}
  \sectionrule{0pt}{0pt}{-5pt}{3pt}}

\newlength{\spacebox}
\settowidth{\spacebox}{8888888888}
\newcommand{\sepspace}{\vspace*{1em}}

\newcommand{\MyName}[1]{
    \Huge \usefont{OT1}{phv}{b}{n} \hfill #1
    \par \normalsize \normalfont}

\newcommand{\MySlogan}[1]{
    \large \usefont{OT1}{phv}{m}{n}\hfill \textit{#1}
    \par \normalsize \normalfont}

\newcommand{\NewPart}[1]{\section*{\uppercase{#1}}}

\newcommand{\PersonalEntry}[2]{
    \noindent\hangindent=2em\hangafter=0
    \parbox{\spacebox}{\textit{#1}}
    \hspace{1.5em} #2 \par}

\newcommand{\SkillsEntry}[2]{
    \noindent\textbf{\textit{#1}} \hspace{1.5em} #2 \par}



\newcommand{\EducationEntry}[4]{
    \noindent \textbf{#1} \hfill {#2} \par
    \noindent \textit{#3} \par
    \noindent \small #4
    \normalsize \par}

\newcommand{\WorkEntry}[4]{
    \noindent \textbf{#1} \hfill {#2} \par
    \noindent \textit{#3} \par
    \noindent \small #4
    \normalsize \par}

\newcommand{\ProjectEntry}[4]{
    \noindent \textbf{#1} \hfill {#2} \par
    \noindent \textit{#3} \par
    \noindent \small #4
    \normalsize \par}

\newcommand{\AwardEntry}[4]{
    \noindent \textbf{#1} \hfill {#2} \par
    \noindent \textit{#3} \par
    \noindent \small #4
    \normalsize \par}

\newcommand{\AboutEntry}[1]{
    \noindent #1 \par}

\begin{document}

% Name and Contact Information
\MyName{Aryasomayajula Ram Bharadwaj}
\bigskip

{\small \hfill \href{mailto:ram.bharadwaj.arya@gmail.com}{ram.bharadwaj.arya@gmail.com} | +91-9108832338 | Bengaluru, India | \href{https://github.com/rokosbasilisk}{github.com/rokosbasilisk}}

% ABOUT Section
\NewPart{ABOUT}
\AboutEntry{Versatile ML Engineer and Backend Developer with expertise in deep learning and strong DevOps knowledge. Skilled in multiple programming languages and refactoring legacy codebases, passionate about AI research.}

% EDUCATION Section
\NewPart{EDUCATION}
\EducationEntry
{B.Tech Electronics and Communications}
{2015 - 2019}
{GMR Institute of Technology, Andhra Pradesh}

% PROFESSIONAL EXPERIENCE Section
\NewPart{PROFESSIONAL EXPERIENCE}

\WorkEntry
{Senior Developer at Innovation \& Development Labs}
{June 2019 - Nov 2024}
{Musigma Business Solutions, Bengaluru}
{%
\begin{itemize}
\item Led development of multiple high-impact projects in LLM operationalization, automated trading, and MLOps.
\item Specialized in backend development, DevOps, and ML engineering across various domains.
\item Received multiple recognitions including Impact Awards and Star Performer of the Team.
\end{itemize}}

% KEY PROJECTS Section (Professional Projects)
\NewPart{KEY PROJECTS}

\ProjectEntry{LLM Agent Platform}{Dec 2023 - Nov 2024}{Team Lead, ML Engineer, Backend Developer}
{%
\begin{itemize}
\item Designed and implemented a semi-autonomous data analysis platform powered by LLM Agents.
\item Developed training strategies for LLM Agents via automatic prompt optimization and integrated Retrieval-Augmented Generation (RAG) support.
\item Implemented evaluation for LLM Agents' results using an ensemble of locally mosted LLMs.
\item Maintained a customized fork of Microsoft's Autogen framework.
\item Successfully deployed for two major clients, assisted client teams in creating use cases.
\end{itemize}}

\sepspace
\ProjectEntry{High Velocity Trading Platform}{2021 - Dec 2023}{Team Lead, Backend Developer, DevOps}
{%
\begin{itemize}
\item Refactored backend code and migrated deployment from bare-metal to Kubernetes.
\item Implemented automated hyperparameter search for ARIMA models.
\item Rewrote legacy trade-signal generation from R to Scala using Akka framework.
\item Enabled near real-time metric calculation and portfolio visualization.
\end{itemize}}

\sepspace
\ProjectEntry{Next-Gen Model Operationalization Platform}{2019 - 2021}{Backend Developer}
{%
\begin{itemize}
\item Developed automatic retraining pipelines for image classification models.
\item Designed and implemented Java microservices for creating and serving Jupyter notebooks.
\item Engineered a microservice for deploying ML models in a DAG structure with canary and blue-green deployment strategies.
\item Created dashboards for monitoring model and resource usage metrics.
\end{itemize}}

% RESEARCH PAPERS Section
\NewPart{RESEARCH PAPERS}

\ProjectEntry{\href{https://github.com/rokosbasilisk/filler_tokens}{Understanding Hidden Computations in Transformer Language Models}}{August 2024}{Independent Researcher}
{%
\begin{itemize}
\item Conducted research to decode the "hidden" computations in transformer models during Chain of Thought (COT) reasoning.
\item Investigated methods to enhance the faithfulness and internal mechanics of chain-of-thought reasoning in LLMs.
\end{itemize}}

\sepspace
\ProjectEntry{\href{https://github.com/rokosbasilisk/platonic-rep}{Investigating the Platonic Representation Hypothesis}}{October 2024}{Independent Researcher}
{%
\begin{itemize}
\item Evaluated representational convergence across in-distribution, out-of-distribution (ImageNet-O), and random noise data, identifying limits of the Platonic Representation Hypothesis. 
\item Conducted noise injection experiments on Vision Transformer models, tracked alignment across LLM training stages, showing phases of similarity that indicate convergence toward a shared structure.
\end{itemize}}

% PERSONAL PROJECTS Section
\NewPart{PERSONAL PROJECTS}

\ProjectEntry{AdvisorMatch}{Nov 2024}{Creator, Developer}
{%
\begin{itemize}
\item Built a PhD advisor-matching tool, AdvisorMatch (\href{https://advisormatch.in}{advisormatch.in}), to help candidates find faculty advisors aligned with their research interests.
\item Designed a similarity search mechanism that matches the user's research interests with the latest work of over 23,000 global faculty to recommend suitable matches.
\end{itemize}}

% SKILLS Section
\NewPart{SKILLS}
\SkillsEntry{Programming Languages:}{Scala, Python, Java, R}
\SkillsEntry{Machine Learning Frameworks:}{PyTorch, JAX}
\SkillsEntry{DevOps Tools:}{Kubernetes, Docker Swarm}

% AWARDS & RECOGNITIONS Section
\NewPart{AWARDS \& RECOGNITIONS}

\AwardEntry{AI Alignment Awards}{July 2023}{www.alignmentawards.com}
{Winning entry for the "goal misgeneralization" track research proposal, selected among 118 entries in this research competition.}

\sepspace

\AwardEntry{Honorable Mention}{Mar 2022}{Alignment Research Center}
{"Eliciting Latent Knowledge" is an open research problem in AI Safety. My proposal was one of the honorable mentions in the ELK (Eliciting Latent Knowledge) research competition.}

\sepspace

\AwardEntry{Impact Award}{October 2022, December 2023}{Musigma Business Solutions}
{Recognized for self-engagement, DevOps skills, and designing the LLM-Agent conversation platform.}

\sepspace

\AwardEntry{Star Performer of the Team}{2019 to 2023}{Musigma Business Solutions}
{Received the SPOT award six times for exceptional contributions.}

\sepspace

\AwardEntry{Bronze Position}{Feb 2021}{Build-on-Redis Hackathon}
{Winning entry in the Build-on-Redis hackathon. Implemented a text2code search tool entirely on Redis Stack using CodeBERT model's embeddings for a private code repository.}

\end{document}
