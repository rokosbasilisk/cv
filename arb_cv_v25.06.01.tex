\documentclass[fontsize=11pt]{article}
\usepackage[english]{babel}
\usepackage[utf8]{inputenc}
\usepackage[T1]{fontenc}
\usepackage{lmodern}
\usepackage[protrusion=true,expansion=true]{microtype}
\usepackage[svgnames]{xcolor}
\usepackage[margin=0.75in]{geometry}
\usepackage{url}
\usepackage{textcomp}
\usepackage{hyperref}

\makeatletter
\def\url@modernstyle{
  \@ifundefined{selectfont}{\def\UrlFont{\sf}}{\def\UrlFont{}}}
\makeatother
\urlstyle{modern}

\frenchspacing
\pagestyle{empty}

\renewcommand{\familydefault}{\sfdefault}

\usepackage{sectsty}
\sectionfont{
  \usefont{OT1}{phv}{b}{n}
  \sectionrule{0pt}{0pt}{-5pt}{3pt}}

\newlength{\spacebox}
\settowidth{\spacebox}{8888888888}
\newcommand{\sepspace}{\vspace*{1em}}

\newcommand{\MyName}[1]{
    \Huge \usefont{OT1}{phv}{b}{n} \hfill #1
    \par \normalsize \normalfont}

\newcommand{\MySlogan}[1]{
    \large \usefont{OT1}{phv}{m}{n}\hfill \textit{#1}
    \par \normalsize \normalfont}

\newcommand{\NewPart}[1]{\section*{\uppercase{#1}}}

\newcommand{\PersonalEntry}[2]{
    \noindent\hangindent=2em\hangafter=0
    \parbox{\spacebox}{\textit{#1}}
    \hspace{1.5em} #2 \par}

\newcommand{\SkillsEntry}[2]{
    \noindent\textbf{\textit{#1}} \hspace{1.5em} #2 \par}



\newcommand{\EducationEntry}[4]{
    \noindent \textbf{#1} \hfill {#2} \par
    \noindent \textit{#3} \par
    \noindent \small #4
    \normalsize \par}

\newcommand{\WorkEntry}[4]{
    \noindent \textbf{#1} \hfill {#2} \par
    \noindent \textit{#3} \par
    \noindent \small #4
    \normalsize \par}

\newcommand{\ProjectEntry}[4]{
    \noindent \textbf{#1} \hfill {#2} \par
    \noindent \textit{#3} \par
    \noindent \small #4
    \normalsize \par}

\newcommand{\AwardEntry}[4]{
    \noindent \textbf{#1} \hfill {#2} \par
    \noindent \textit{#3} \par
    \noindent \small #4
    \normalsize \par}

\newcommand{\AboutEntry}[1]{
    \noindent #1 \par}

\begin{document}

% Name and Contact Information
\MyName{Aryasomayajula Ram Bharadwaj}
\bigskip

{\small \hfill \href{mailto:ram.bharadwaj.arya@gmail.com}{ram.bharadwaj.arya@gmail.com} | +91-9108832338 | Bengaluru, India | \href{https://github.com/rokosbasilisk}{github.com/rokosbasilisk}}

% ABOUT Section
\NewPart{ABOUT}
\AboutEntry{Versatile ML Engineer and Backend Developer with expertise in deep learning and strong DevOps knowledge. Skilled in multiple programming languages and refactoring legacy codebases, passionate about AI research.}

% EDUCATION Section
\NewPart{EDUCATION}
\EducationEntry
{B.Tech Electronics and Communications}
{2015 - 2019}
{GMR Institute of Technology, Andhra Pradesh}

% PROFESSIONAL EXPERIENCE Section
\NewPart{PROFESSIONAL EXPERIENCE}
\WorkEntry
{AI Resident - Lossfunk AI Residency}
{April 2025 - May 2025}
{Lossfunk AI Residency (Remote)}
{%
\begin{itemize}
\item Selected for Lossfunk, a 6-week intensive AI residency with 10 highly skilled builders and researchers.
\item Developed \href{https://github.com/rokosbasilisk/STU-PID}{STU-PID (Steering Token Usage with PID Control)}, a novel activation steering technique to reduce redundant reasoning tokens in LLMs.
\item Conducted experiments on reasoning models like DeepSeek-R1-Distill-Qwen-1.5B with dynamic activation interventions, achieving 32\% token reduction and improved reasoning accuracy on grade school math problems.
\item A detailed paper on this is accessible at \href{https://arxiv.org/abs/2506.18831}{Steering Token Usage with PID Control} .
\end{itemize}}

\WorkEntry
{Associate Technical Architect - MLOps}
{Nov 2024 - Present}
{Quantiphi Analytics, Bengaluru}
{%
\begin{itemize}
\item Designed and implemented an AI agent system for automated replies for a major telecom company's sales chatbot.
\item Experimented with LangGraph agent architectures to improve controllability and modularity in conversational flows.
\item Revamped and automated the scheduled RAG data ingestion pipeline and Improved retrieval accuracy by 20\%.
\item Improved retriever latency by several folds via extensive code refactoring.
\end{itemize}}
\WorkEntry
{Senior Developer at Innovation \& Development Labs}
{June 2019 - Nov 2024}
{Musigma Business Solutions, Bengaluru}
{%
\begin{itemize}
\item Led development of multiple high-impact projects in LLM operationalization, automated trading, and MLOps.
\item Specialized in backend development, DevOps, and ML engineering across various domains.
\item Received multiple recognitions including Impact Awards and Star Performer of the Team.
\end{itemize}}

% KEY PROJECTS Section (Professional Projects)\
\NewPart{KEY PROJECTS}

\ProjectEntry{LLM Agent Platform}{Dec 2023 - Nov 2024}{Team Lead, ML Engineer, Backend Developer}
{%
\begin{itemize}
\item Designed and implemented a semi-autonomous data analysis platform powered by LLM Agents.
\item Developed training strategies for LLM Agents via automatic prompt optimization and integrated Retrieval-Augmented Generation (RAG) support.
\item Implemented evaluation for LLM Agents' results using an ensemble of locally mosted LLMs.
\item Maintained a customized fork of Microsoft's Autogen framework.
\item Successfully deployed for two major clients, assisted client teams in creating use cases.
\end{itemize}}

\newpage
\ProjectEntry{High Velocity Trading Platform}{2021 - Dec 2023}{Team Lead, Backend Developer, DevOps}
{%
\begin{itemize}
\item Refactored backend code and migrated deployment from bare-metal to Kubernetes.
\item Implemented automated hyperparameter search for ARIMA models.
\item Rewrote legacy trade-signal generation from R to Scala using Akka framework.
\item Enabled near real-time metric calculation and portfolio visualization.
\end{itemize}}

\sepspace
\ProjectEntry{Next-Gen Model Operationalization Platform}{2019 - 2021}{Backend Developer}
{%
\begin{itemize}
\item Developed automatic retraining pipelines for image classification models.
\item Designed and implemented Java microservices for creating and serving Jupyter notebooks.
\item Engineered a microservice for deploying ML models in a DAG structure with canary and blue-green deployment strategies.
\end{itemize}}

% RESEARCH PAPERS Section
\NewPart{RESEARCH PAPERS}

\ProjectEntry{\href{https://arxiv.org/html/2412.04537}{Understanding Hidden Computations in Transformer Language Models}}{August 2024}{Independent Researcher}
{%
\begin{itemize}
\item Conducted research to decode the "hidden" computations in transformer models during Chain of Thought (COT) reasoning.
\item Investigated methods to enhance the faithfulness and internal mechanics of chain-of-thought reasoning in LLMs.
\end{itemize}}


% PERSONAL PROJECTS Section
\NewPart{PERSONAL PROJECTS}
\ProjectEntry{\href{https://wiserank.io}{Wiserank.io}}{June 2025}{Creator, Developer}
{%
\begin{itemize}
\item Built 'Wiserank.io', a research paper search engine, to rank and retrieve research papers based on relevance and novelty.
\item Implemented a feature to automatically retrieve and generate relevant citations for an uploaded research manuscript.
\end{itemize}}

% SKILLS Section
\NewPart{SKILLS}
\SkillsEntry{Programming Languages:}{Scala, Python, Java, R}
\SkillsEntry{Machine Learning Frameworks:}{PyTorch, JAX}

% AWARDS & RECOGNITIONS Section
\NewPart{AWARDS \& RECOGNITIONS}

\AwardEntry{AI Alignment Awards}{July 2023}{www.alignmentawards.com}
{Winning entry for the "goal misgeneralization" track research proposal, selected among 118 entries in this research competition.}

\sepspace

\AwardEntry{Honorable Mention}{Mar 2022}{Alignment Research Center}
{"Eliciting Latent Knowledge" is an open research problem in AI Safety. My proposal was one of the honorable mentions in this research competition.}

\AwardEntry{SPOT and Impact awards}{2019 - 2023}{Musigma Business Solutions}
{Received six SPOT (Star Performer) awards and two Impact Awards at Musigma for exceptional contributions, DevOps skills, and LLM-Agent platform design.}

\sepspace

\AwardEntry{Bronze Position}{Feb 2021}{Build-on-Redis Hackathon}
{Winning entry in the Build-on-Redis hackathon. Implemented a text2code search tool entirely on Redis Stack using CodeBERT model's embeddings for a private code repository.}

\end{document}
