\documentclass[10pt,letterpaper]{article}
\usepackage[utf8]{inputenc}
\usepackage[T1]{fontenc}
\usepackage{lmodern}
\usepackage[margin=0.5in]{geometry}
\usepackage{multicol}
\usepackage{enumitem}
\usepackage{titlesec}

\pagestyle{empty}
\setlength{\parindent}{0pt}

% Custom section command
\titleformat{\section}{\large\bfseries\uppercase}{}{0em}{}[\titlerule]
\titlespacing*{\section}{0pt}{*1}{*1}

% Two-column layout
\setlength{\columnsep}{0.5cm}

% Compact list settings
\setlist[itemize]{leftmargin=*, nosep, topsep=0pt, partopsep=0pt, parsep=0pt}

% Custom commands
\newcommand{\entry}[4]{
    \textbf{#1} \hfill #2 \\
    \textit{#3} \\
    #4
    \vspace{0.3em}
}

\begin{document}

\begin{center}
    \huge\textbf{Aryasomayajula Ram Bharadwaj} \\
    \small ram.bharadwaj.arya@gmail.com | 9108832338 | Benguluru, India | github.com/rokosbasilisk \\
    \textit{Passionate ML Engineer and Backend Developer with a strong research aptitude}
\end{center}

\vspace{-0.5em}

\begin{multicols}{2}

\section{Experience}
\entry{ML Engineer}{2019 - Present}{Musigma Business Solutions, Benguluru}{
    \begin{itemize}
        \item Designed and implemented semi-autonomous LLM Agent platform for data-analysis
        \item Implemented RAG support and prompt optimization for LLM agents
        \item Evaluated hyperparameter optimization algorithms for timeseries models
        \item Designed automated intra-day trading system
        \item Implemented auto-hyperparameter tuning for time series models
    \end{itemize}
}

\entry{Backend Developer}{2019 - Present}{Musigma Business Solutions, Benguluru}{
    \begin{itemize}
        \item Implemented backend stack for LLM Agent platform
        \item Maintained internal fork of Microsoft's Autogen framework
        \item Migrated legacy code from R to Scala
        \item Designed deployment strategies for ML models on Kubernetes
        \item Received "Star performer" 6 times and "Impact Award" twice
    \end{itemize}
}

\entry{ML Alignment Theory Research Trainee}{May - July 2023}{Stanford Existential Risk Initiative, Remote}{
    \begin{itemize}
        \item Underwent AI alignment research training under John Wentworth
    \end{itemize}
}

\section{Education}
\entry{B.Tech Electronics and Communications}{2015 - 2019}{GMR Institute of Technology, Andhra Pradesh}{CGPA: 7.53/10}

\section{Projects}
\entry{LLM Agent platform}{}{flask, autogen}{
    Automated data science tasks platform using autogen, deployed for internal use.
}

\entry{Tplusone.Org}{}{}{
    Web application for financial market analysis and algorithmic trading visualization.
}

\entry{In-house MLOps platform}{}{}{
    End-to-end platform for deploying and monitoring ML models.
}

\columnbreak

\entry{Synthetic Debate Data Generation}{}{python, transformers}{
    Platform for generating debate transcripts using GPT-4, benchmarked mistral and pythia models via LoRA finetuning.
}

\entry{Opensource-contributions}{}{Python}{
    Framework for evaluating LLM-generated code in sandboxed environment.
}

\entry{Telugu text recognition}{}{python, torch}{
    CRNN for recognizing strings of Telugu handwritten characters.
}

\section{Skills}
\textbf{Languages:} Scala, Python, Java, R \\
\textbf{ML:} PyTorch, JAX \\
\textbf{DevOps:} Kubernetes, Docker-swarm \\
\textbf{Misc:} Akka, Spring-boot \\
\textbf{AI/ML:} LLM Finetuning, Transformer architectures, Deep neural networks

\section{Awards}
\entry{AI Alignment Awards}{July 2023}{www.alignmentawards.com}{
    Winning entry for "goal misgeneralization" track
}

\entry{Honorable Mention}{Mar 2022}{Alignment Research Center}{
    For ELK (Eliciting Latent Knowledge) research proposal
}

\entry{Impact Award}{Oct 2022, Dec 2023}{Musigma Business Solutions}{
    For exceptional work and implementing LLM-Agent conversation platform
}

\entry{Bronze-position}{Feb 2021}{Build-on-Redis Hackathon}{
    Built natural-language semantic code-search using codeBERT
}

\end{multicols}

\end{document}
