\documentclass[fontsize=11pt]{article}
\usepackage[english]{babel}
\usepackage[utf8]{inputenc}
\usepackage[T1]{fontenc}
\usepackage{lmodern}
\usepackage[protrusion=true,expansion=true]{microtype}
\usepackage[svgnames]{xcolor}
\usepackage[margin=0.75in]{geometry}
\usepackage{url}
\usepackage{textcomp}
\usepackage{hyperref}

\makeatletter
\def\url@modernstyle{
  \@ifundefined{selectfont}{\def\UrlFont{\sf}}{\def\UrlFont{}}}
\makeatother
\urlstyle{modern}

\frenchspacing
\pagestyle{empty}

\renewcommand{\familydefault}{\sfdefault}

\usepackage{sectsty}
\sectionfont{
  \usefont{OT1}{phv}{b}{n}
  \sectionrule{0pt}{0pt}{-5pt}{3pt}}

\newlength{\spacebox}
\settowidth{\spacebox}{8888888888}
\newcommand{\sepspace}{\vspace*{1em}}

\newcommand{\MyName}[1]{
    \Huge \usefont{OT1}{phv}{b}{n} \hfill #1
    \par \normalsize \normalfont}

\newcommand{\MySlogan}[1]{
    \large \usefont{OT1}{phv}{m}{n}\hfill \textit{#1}
    \par \normalsize \normalfont}

\newcommand{\NewPart}[1]{\section*{\uppercase{#1}}}

\newcommand{\PersonalEntry}[2]{
    \noindent\hangindent=2em\hangafter=0
    \parbox{\spacebox}{\textit{#1}}
    \hspace{1.5em} #2 \par}

\newcommand{\SkillsEntry}[2]{
    \noindent\textbf{\textit{#1}} \hspace{1.5em} #2 \par}

\newcommand{\EducationEntry}[4]{
    \noindent \textbf{#1} \hfill {#2} \par
    \noindent \textit{#3} \par
    \noindent \small #4
    \normalsize \par}

\newcommand{\WorkEntry}[4]{
    \noindent \textbf{#1} \hfill {#2} \par
    \noindent \textit{#3} \par
    \noindent \small #4
    \normalsize \par}

\newcommand{\ProjectEntry}[4]{
    \noindent \textbf{#1} \hfill {#2} \par
    \noindent \textit{#3} \par
    \noindent \small #4
    \normalsize \par}

\newcommand{\AwardEntry}[4]{
    \noindent \textbf{#1} \hfill {#2} \par
    \noindent \textit{#3} \par
    \noindent \small #4
    \normalsize \par}

\newcommand{\AboutEntry}[1]{
    \noindent #1 \par}

\begin{document}

% Name and Contact Information
\MyName{Aryasomayajula Ram Bharadwaj}
\bigskip

{\small \hfill \href{mailto:ram.bharadwaj.arya@gmail.com}{ram.bharadwaj.arya@gmail.com} | +91-9108832338 | Bengaluru, India | \href{https://github.com/rokosbasilisk}{github.com/rokosbasilisk}}

% PROFESSIONAL SUMMARY Section
\NewPart{PROFESSIONAL SUMMARY}
\AboutEntry{Senior ML Engineer with 6+ years developing production AI systems and leading engineering teams. Background in AI safety research with published work on model interpretability and LLM inference optimization. Experienced architecting scalable solutions for conversational AI and agentic workflows. Strong expertise in MLOps, distributed systems, and technical leadership.}

% EDUCATION Section
\NewPart{EDUCATION}
\EducationEntry
{Bachelor of Technology - Electronics and Communications Engineering}
{2015 - 2019}
{GMR Institute of Technology, Andhra Pradesh}
{}

% SKILLS Section
\NewPart{SKILLS}
\SkillsEntry{Programming:}{Python, Scala, Java, R, SQL}
\SkillsEntry{ML/AI:}{PyTorch, JAX, Transformers, LangChain, AutoGen}
\SkillsEntry{DevOps:}{Docker, Kubernetes, CI/CD, GCP}
\SkillsEntry{Systems:}{Redis, Kafka, Microservices}

% PROFESSIONAL EXPERIENCE Section
\NewPart{PROFESSIONAL EXPERIENCE}
\WorkEntry
{Associate Technical Architect - MLOps}
{Nov 2024 - Present}
{Quantiphi Analytics, Bengaluru}
{%
\begin{itemize}
\item Designed and implemented an AI agent system for automated replies for a telecom client's sales chatbot
\item Experimented with LangGraph agent architectures to improve controllability and modularity in conversational flows
\item Revamped and automated the scheduled RAG data ingestion pipeline and improved retrieval accuracy by 20\%
\item Improved retriever latency from 6 seconds to 1.4 seconds through extensive code refactoring
\end{itemize}}

\WorkEntry
{AI Resident - Lossfunk AI Residency}
{April 2025 - May 2025}
{Lossfunk AI Residency (Remote)}
{%
\begin{itemize}
\item Selected among 100 applicants for elite 6-week intensive AI residency program with 10 researchers
\item Developed \href{https://github.com/rokosbasilisk/STU-PID}{STU-PID (Steering Token Usage with PID Control)}, a novel activation steering technique to reduce redundant reasoning tokens in LLMs
\item Conducted experiments on reasoning models like DeepSeek-R1-Distill-Qwen-1.5B with dynamic activation interventions, achieving 32\% token reduction and improved reasoning accuracy on GSM8K benchmark
\item Published research findings: \href{https://arxiv.org/abs/2506.18831}{Steering Token Usage with PID Control}
\end{itemize}}

\WorkEntry
{Senior Developer - Innovation \& Development Labs}
{June 2019 - Nov 2024}
{Musigma Business Solutions, Bengaluru}
{%
\begin{itemize}
\item Led development of multiple high-impact projects in LLM operationalization, automated trading, and MLOps
\item Specialized in backend development, DevOps, and ML engineering across various domains
\item Received multiple recognitions including Impact Awards and Star Performer of the Team
\end{itemize}}



% KEY PROJECTS Section
\NewPart{KEY TECHNICAL PROJECTS}

\ProjectEntry{LLM Agent Platform}{Dec 2023 - Nov 2024}{Team Lead, ML Engineer, Backend Developer}
{%
\begin{itemize}
\item Designed and implemented semi-autonomous data analysis platform using Microsoft AutoGen framework
\item Developed automated prompt optimization strategies and integrated RAG support
\item Built evaluation framework using ensemble of locally hosted LLMs
\item Successfully deployed for two major clients, assisted client teams in creating use cases
\end{itemize}}

\ProjectEntry{High-Velocity Trading Platform}{2021 - Dec 2023}{Team Lead, Backend Developer, DevOps}
{%
\begin{itemize}
\item Refactored backend code and migrated deployment from bare-metal to Kubernetes
\item Implemented automated hyperparameter search for ARIMA models
\item Rewrote legacy trade-signal generation from R to Scala using Akka framework
\item Enabled near real-time metric calculation and portfolio visualization
\end{itemize}}

\ProjectEntry{ML Model Operationalization Platform}{2019 - 2021}{Backend Developer}
{%
\begin{itemize}
\item Developed automatic retraining pipelines for image classification models
\item Designed Java microservices for creating and serving Jupyter notebooks
\item Engineered ML model deployment service with canary and blue-green deployment strategies
\end{itemize}}

% RESEARCH \& PUBLICATIONS Section
\NewPart{RESEARCH \& PUBLICATIONS}

\ProjectEntry{\href{https://arxiv.org/abs/2506.18831}{Steering Token Usage with PID Control}}{June 2025}{Lead Author}
{Novel technique reducing computational overhead in LLMs through activation steering with 32\% token reduction on GSM8K.}

\ProjectEntry{\href{https://arxiv.org/html/2412.04537}{Understanding Hidden Computations in Transformer Language Models}}{August 2024}{Lead Author}
{Investigated internal mechanisms of chain-of-thought reasoning and developed interpretability methods for LLM reasoning.}

% PERSONAL PROJECTS Section
\NewPart{SELECTED PERSONAL PROJECTS}
\ProjectEntry{\href{https://wiserank.io}{Wiserank.io - AI-Powered Research Discovery}}{June 2025}{Creator \& Full-Stack Developer}
{%
\begin{itemize}
\item Built intelligent research paper search engine with relevance ranking and novelty detection algorithms
\item Implemented automatic citation generation feature for uploaded manuscripts using NLP and knowledge graphs
\item Deployed scalable architecture serving researchers with enhanced paper discovery and citation workflows
\end{itemize}}

% AWARDS & RECOGNITIONS Section
\NewPart{AWARDS \& RECOGNITIONS}

\AwardEntry{AI Alignment Awards - Winner}{July 2023}{\href{https://www.lesswrong.com/posts/zFoAAD7dfWdczxoLH/winners-of-ai-alignment-awards-research-contest}{AI Safety Research Competition}}
{Selected among 118 global entries for winning research proposal on "goal misgeneralization" in AI systems.}

\AwardEntry{Honorable Mention - Eliciting Latent Knowledge}{March 2022}{Alignment Research Center}
{Recognized for innovative approach to open research problem in AI interpretability and knowledge extraction.}

\AwardEntry{Bronze Medal - Build-on-Redis Hackathon}{February 2021}{Redis Labs}
{Developed text-to-code search tool using CodeBERT embeddings and Redis Stack for private repository indexing.}

\AwardEntry{Excellence Awards (8x)}{2019 - 2023}{Musigma Business Solutions}
{6 SPOT (Star Performer) awards and 2 Impact Awards for technical leadership, innovation, and delivery excellence.}

\end{document}
