\documentclass[11pt]{article}
\usepackage[margin=0.6in]{geometry}
\usepackage{hyperref}
\usepackage{enumitem}
\usepackage{titlesec}
\usepackage{array}
\pagestyle{empty}

\titleformat{\section}{\large\bfseries\uppercase}{}{0em}{}[\titlerule]
\titlespacing{\section}{0pt}{8pt}{4pt}

\begin{document}

\begin{center}
\textbf{\Large Aryasomayajula Ram Bharadwaj} \\
\vspace{4pt}
Bengaluru, India | \href{mailto:ram.bharadwaj.arya@gmail.com}{ram.bharadwaj.arya@gmail.com} | +91-9108832338 | \href{https://github.com/rokosbasilisk}{github.com/rokosbasilisk}
\end{center}

\vspace{12pt}

\section{Profile}
Research-focused AI Engineer working at the intersection of agentic workflows, interpretability, and AI safety. Six years of experience in ML infrastructure, workflow automation, and large-model analysis. Published papers on representation scaling, activation steering, and reasoning efficiency. Skilled at bridging deep research with deployable production systems.

\section{Core Competencies}
\begin{itemize}[leftmargin=*,nosep]
\item \textbf{Research:} Mechanistic interpretability, scaling laws, activation steering, representation alignment, model awareness
\item \textbf{Frameworks:} PyTorch, JAX, Transformers, LangGraph, AutoGen
\item \textbf{Systems:} Docker, Kubernetes, Redis, Kafka, PostgreSQL, Microservices
\item \textbf{Cloud:} AWS, GCP
\item \textbf{Languages:} Python, Scala, Java
\end{itemize}

\section{Research \& Publications}
\textbf{\href{https://fullwrong.com/2025/07/23/scaling-compression/}{Scaling Laws for LLM-Based Data Compression}} \hfill \textbf{Jul 2025}\\
\textit{Lead Investigator}
\begin{itemize}[leftmargin=*,nosep]
\item Derived universal power-law scaling between model size and compression efficiency across text, image, and speech domains.
\end{itemize}

\textbf{\href{https://arxiv.org/abs/2506.18831}{Steering Token Usage with PID Control}} \hfill \textbf{Jun 2025}\\
\textit{Lead Investigator}
\begin{itemize}[leftmargin=*,nosep]
\item Proposed a redundancy-aware activation steering method that reduced reasoning token usage by 32\% while improving GSM8K accuracy.
\item Released open-source implementation \href{https://github.com/rokosbasilisk/STU-PID}{STU-PID}.
\end{itemize}

\textbf{\href{https://arxiv.org/html/2412.04537}{Understanding Hidden Computations in Transformer Language Models}} \hfill \textbf{Aug 2024}\\
\textit{Lead Investigator}
\begin{itemize}[leftmargin=*,nosep]
\item Analyzed an open problem underlying interpretability of chain-of-thought reasoning.
\item Designed probing techniques revealing latent computation patterns in transformer intermediate layers.
\end{itemize}

\section{Independent Projects}
\textbf{\href{https://wiserank.io}{Wiserank.io – AI-Powered Research Discovery}} \hfill \textbf{Jun 2025}\\
\textit{Creator \& Full-Stack Developer}
\begin{itemize}[leftmargin=*,nosep]
\item Built research-paper search engine ranking works by originality and citation impact; integrated automatic reference generation and semantic vector search.
\item Achieved 100+ active research users in first few months.
\end{itemize}

\section{Awards \& Recognition}
\textbf{Winner – AI Alignment Awards} \hfill \textbf{Jul 2023}\\
Selected among 118 global entries for work on goal misgeneralization and alignment failure modes.

\textbf{Honorable Mention – Eliciting Latent Knowledge (ARC)} \hfill \textbf{Mar 2022}\\
Recognized for novel approach to eliciting implicit model knowledge.

\textbf{Bronze Medal – Build-on-Redis Hackathon} \hfill \textbf{Feb 2021}\\
Developed private code search engine using CodeBERT embeddings and Redis Stack.

\textbf{Excellence Awards (8×) – MuSigma Business Solutions} \hfill \textbf{2019–2023}

\newpage
\section{Professional Experience}

\textbf{Associate Technical Architect – Platform} \hfill \textbf{Nov 2024 – Present}\\
\textit{Quantiphi Analytics, Bengaluru}
\begin{itemize}[leftmargin=*,nosep]
\item Designed AI agent system for automated issue-severity classification and escalation management.
\item Architected and implemented a conversational AI-agent chatbot for answering sales-related queries at a major telecom company, refactoring legacy systems to a leaner modular implementation.
\item Revamped the scheduled RAG ingestion pipeline to improve retrieval accuracy and reduce time-to-first-token by 3×.
\end{itemize}

\textbf{AI Resident – Lossfunk AI Residency} \hfill \textbf{May 2025}\\
\textit{Lossfunk Research Residency}
\begin{itemize}[leftmargin=*,nosep]
\item Selected among 100 applicants for elite 6-week intensive AI residency program with 10 researchers.
\item Developed redundancy-aware steering techniques (STU-PID) achieving 32\% token reduction and improved reasoning accuracy on GSM8K.
\end{itemize}

\textbf{ML Engineer – Innovation \& Development Labs} \hfill \textbf{Jun 2019 – Nov 2024}\\
\textit{MuSigma Business Solutions, Bengaluru}
\begin{itemize}[leftmargin=*,nosep]
\item Led development of multiple high-impact projects in LLM operationalization, automated trading, and MLOps.
\item Specialized in backend development, DevOps, and ML engineering across multiple domains.
\item Refactored backend code and migrated trading deployments from bare-metal to Kubernetes with automated hyperparameter search for ARIMA models.
\item Rewrote legacy trade-signal generation system from R to Scala (Akka framework), enabling near-real-time metric computation and visualization.
\item Designed ML model deployment microservices with canary and blue-green deployment strategies.
\end{itemize}

\section{Key Projects}
\textbf{Conversational Sales Chatbot using AI Agents} \hfill \textbf{2024 – Present}\\
\begin{itemize}[leftmargin=*,nosep]
\item Designed and implemented an AI-agent-based system for automated replies for a telecom client's sales chatbot.
\item Refactored codebase from a proprietary workflow engine to LangGraph, reducing code size and maintenance complexity.
\end{itemize}

\textbf{LLM Agent Platform} \hfill \textbf{2023–2024}\\
\begin{itemize}[leftmargin=*,nosep]
\item Designed a semi-autonomous data-analysis platform using AutoGen.
\item Integrated automated prompt optimization strategies and RAG evaluation support with ensembles of locally hosted LLMs.
\end{itemize}

\textbf{High-Velocity Trading Platform} \hfill \textbf{2021–2023}\\
\begin{itemize}[leftmargin=*,nosep]
\item Refactored backend and enabled migration from legacy R-based signal engine to Scala using Akka.
\item Added automated retraining and portfolio metrics modules with visualization dashboards.
\end{itemize}

\textbf{ML Model Operationalization Platform} \hfill \textbf{2019–2021}\\
\begin{itemize}[leftmargin=*,nosep]
\item Built microservice-based pipelines for retraining and deploying ML models with CI/CD integration.
\item Automated image-classification retraining workflows with configurable deployment templates.
\end{itemize}

\section{Education}
\textbf{Bachelor of Technology – Electronics and Communications Engineering} \hfill \textbf{2015–2019}\\
GMR Institute of Technology, Andhra Pradesh

\end{document}

